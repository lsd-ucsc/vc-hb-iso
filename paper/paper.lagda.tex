\documentclass[acmsmall,review,anonymous]{acmart}
\settopmatter{printfolios=true,printccs=false,printacmref=false}

%% Remove copyright box
\setcopyright{none}
\renewcommand\footnotetextcopyrightpermission[1]{}
\pagestyle{plain}

\bibliographystyle{ACM-Reference-Format}
\citestyle{acmauthoryear}

\usepackage{newunicodechar}
\newunicodechar{ℕ}{\ensuremath{\mathbb{N}}}
\newunicodechar{⊏}{\ensuremath{\sqsubset}}
\newunicodechar{⊎}{\ensuremath{\uplus}}
\newunicodechar{⊔}{\ensuremath{\sqcup}}
\newunicodechar{≺}{\ensuremath{\prec}}

\usepackage{amsthm}
\theoremstyle{definition}
\newtheorem{definition}{Definition}
\theoremstyle{theorem}
\newtheorem{throrem}{Throrem}

\usepackage[references]{agda}

\begin{code}[hide]
module paper where

open import Data.Empty using (⊥-elim)
open import Data.Fin using (Fin)
open import Data.Nat as ℕ
open import Data.Nat.Properties as ℕₚ
open import Data.Sum as Sum using (_⊎_; inj₁; inj₂)
open import Data.Product using (_×_; _,_; proj₁; proj₂; ∃-syntax)
open import Data.Vec.Functional as Vector hiding (init)
open import Data.Vec.Functional.Properties as Vectorₚ
open import Data.Vec.Functional.Relation.Binary.Pointwise using (Pointwise)
open import Function using (const)
open import Relation.Binary using (tri<; tri≈; tri>)
open import Relation.Binary.HeterogeneousEquality using (_≅_; refl)
open import Relation.Binary.PropositionalEquality using (_≡_; refl)
open import Relation.Nullary using (¬_; yes; no)
\end{code}

\title{Formalizing Vector Clocks in Agda}

\author{Gan Shen}
\affiliation{\institution{University of California, Santa Cruz} \country{USA}}
\author{Simon Guo}
\affiliation{\institution{University of California, Santa Cruz} \country{USA}}
\author{Lindsey Kuper}
\affiliation{\institution{University of California, Santa Cruz} \country{USA}}

\begin{abstract}
  TBD
\end{abstract}

\begin{document}

\maketitle

\section{Introduction}
A distributed system is a group of connected computers that are
programmed to do a specific task; these computers are usually
spatially separated and communicate only by exchanging messages. By
having multiple computers located at different places working
together, it increases system throughput, data locality, as well as
fault tolerance. However, it also makes reasoning about programs
harder in that our common persception that events happen in a total
order no longer hold---events can be concurrent and it is not obvious
how to order events from different processes.

One widely used order proposed to solve this problem is Leslie
Lamport's happens-before relation~\cite{lamport1978}, or (potentially)
causality relation because it tracks (potential) causality between
events.

\section{System Model}
We model a distributed system as consisting of a fixed number of
processes communicating solely by messages. In Agda, we postulate the
number of processes \AgdaRef{n}\footnote{\ensuremath{\mathbb{N}} is
the Agda type of natural numbers.} and the type of the messages
\AgdaRef{Message}\footnote{Set is the Agda type of arbitrary types}:
\begin{code}
postulate
  n : ℕ
  Message : Set
\end{code}
In an abstract way, the history of a process can be viewed as a
sequence of events that take place on it, where events are sendings
and receivings of messages. Similarly, the history of one execution of
a distributed system can be viewed as an collection of sequences of
events of each process. We assume each process is assigned an unique
identifier \AgdaRef{ProcessId}\footnote{Fin n is the Agda type of
natural numbers less than n.} and each event is assigned a local
identifier \AgdaRef{LocalEventId} that corresponds to the order it
takes place on its originating process:
\begin{code}
ProcessId = Fin n
LocalEventId = ℕ
\end{code}
\noindent As as result, we can identify
a event by a product of its originating process's \AgdaRef{ProcessId}
and its \AgdaRef{LocalEventId}.

Events are related, one relation that we are in particular interested
in is Lamport's happens-before relation that establishes an strict
partial order on events:
\begin{definition}[Happens-Before]
  We say event $e$ happens before $e'$ if and only if:
  \begin{enumerate}
  \item $e$ and $e'$ occur on the same process and $e$ takes place before $e'$; or
  \item $e$ is a sending event and $e'$ is the correspoinding receiving event; or
  \item $e$ happens before $e''$ and $e''$ happens before $e'$ for any event $e''$.
  \end{enumerate}
\end{definition}
\noindent Two distinct events $e$ and $e'$ are said to be concurrent
if neither $e$ happens before or $e'$ happens before $e$.  It is
well-known that the happens-before relation captures the potential
causality of events in an execution.

To decide if event $e$ happens before $e'$, one has to position the
two events into a particular history and see if one of the three
conditions of happens-before holds.  This requires us to define
reachable histories, which is traditionally done as a state transition
system. Here we present a novel representation of events that carry
the histories they preceive, what we call history-carrying events, so
we can avoid the much heavyweight transition system:
\begin{code}
data Event : ProcessId → LocalEventId → Set where
  init  : ∀ {pid} → Event pid zero
  send  : ∀ {pid} {eid} → Message → Event pid eid → Event pid (suc eid)
  recv  : ∀ {pid pid′} {eid eid′} → Event pid′ eid′ → Event pid eid → Event pid (suc eid)
\end{code}
\noindent \AgdaRef{Event} is indexed by a product of
\AgdaRef{ProcessId} and \AgdaRef{LocalEventId} as its
identifier, and has three constructors:
\begin{enumerate}
\item \AgdaRef{init} is the initial event of each process.
\item \AgdaRef{send} takes a message and a local preceding event, then
  contruct a new event with \AgdaRef{LocalEventId} plus one.
\item \AgdaRef{recv} takes a sending event and a local preceding
  event, then construct a new event with \AgdaRef{LocalEventId} plus
  one.
\end{enumerate}

\begin{code}[hide]
{-# INJECTIVE Event #-}
\end{code}

We define some generalized variables:
\begin{code}
variable
  pid pid′ pid″ : ProcessId
  eid eid′ eid″ : LocalEventId
  m  : Message
  e  : Event pid  eid
  e′ : Event pid′ eid′
  e″ : Event pid″ eid″
\end{code}

We define two auxilliary functions to extrac the type-level
\AgdaRef{ProcessId} and \AgdaRef{LocalEventId} from a \AgdaRef{Event}:
\begin{code}
pid[_] : Event pid eid → ProcessId
pid[_] {pid} {_} _ = pid

eid[_] : Event pid eid → LocalEventId
eid[_] {_} {eid} _ = eid
\end{code}

One problem of our representation of events is that we can talk about
events from seperate executions, to constrain events to be from the
same exectuion, we postulate:
\begin{code}
postulate
  uniquely-identify : pid[ e ] ≡ pid[ e′ ] → eid[ e ] ≡ eid[ e′ ] → e ≅ e′
\end{code}

With this tree-like structure, we can easily define the happens-before
on events as:
\begin{code}
data _⊏_ : Event pid eid → Event pid′ eid′ → Set where
  processOrder₁  : e ⊏ send m e
  processOrder₂  : e ⊏ recv e′ e
  send⊏recv      : e ⊏ recv e  e′
  trans          : e ⊏ e′ → e′ ⊏ e″ → e ⊏ e″
\end{code}

$\sqsubset$ is a strict partial order, that is, it is irreflexive,
transitive, and asymmetric:
\begin{code}
⊏-irrefl : ¬ e ⊏ e
⊏-trans : e ⊏ e′ → e′ ⊏ e″ → e ⊏ e″
⊏-asym : e ⊏ e′ → ¬ e′ ⊏ e
\end{code}

One useful lemma we use a lot in the following sections is that $\sqsubset$ is trichotomous locally, that is, for any two events $e$ and $e'$ from the same process, either $e$ happens before $e'$, or $e$ and $e'$ are equal, or $e'$ hapeens before $e$:
\begin{code}
⊏-tri-locally : pid[ e ] ≡ pid[ e′ ] → e ⊏ e′ ⊎ e ≅ e′ ⊎ e′ ⊏ e
\end{code}

\section{Vector Clock}
We define vector clocks as:
\begin{code}
VC : Set
VC = Vector ℕ n
\end{code}

To compute the vector clock of an event:
\begin{code}
vc[_] : Event pid eid → VC
vc[_] {pid} init        = updateAt pid (const 1) (replicate 0)
vc[_] {pid} (send _ e)  = updateAt pid suc vc[ e ]
vc[_] {pid} (recv e e′) = updateAt pid suc (zipWith _⊔_ vc[ e ] vc[ e′ ])
\end{code}

Vector clock comparison:
\begin{code}
_≺_ : VC → VC → Set
vc ≺ vc′ = Pointwise _≤_ vc vc′ × ∃[ pid ] vc pid < vc′ pid
\end{code}

Vector clocks captures the happens-before relation:
\begin{code}
postulate
  ⊏-preserving : e ⊏ e′ →  vc[ e ] ≺ vc[ e′ ]
  ⊏-determining : vc[ e ] ≺ vc[ e′ ] → e ⊏ e′
\end{code}

\section{Conclusion}
placeholder~\citep{mattern-vector-time, fidge-vector-time, schmuck-dissertation}

\bibliography{refs}

\appendix

\begin{code}
size : Event pid eid → ℕ
size init        = zero
size (send _ e)  = suc (size e)
size (recv e e′) = suc (size e + size e′)

⊏⇒size< : e ⊏ e′ → size e < size e′
⊏⇒size< processOrder₁ = s≤s ≤-refl
⊏⇒size< processOrder₂ = s≤s (≤-stepsˡ _ ≤-refl)
⊏⇒size< send⊏recv     = s≤s (≤-stepsʳ _ ≤-refl)
⊏⇒size< (trans x y)   = ≤-trans (⊏⇒size< x) (<⇒≤ (⊏⇒size< y))

⊏-irrefl x = 1+n≰n (⊏⇒size< x)

⊏-trans = trans

⊏-asym x y = ⊥-elim (⊏-irrefl (⊏-trans x y))

eid<⇒⊏-locally : pid[ e ] ≡ pid[ e′ ] → eid[ e ] < eid[ e′ ] → e ⊏ e′
eid<⇒⊏-locally {_} {eid} {e} {_} {suc eid′} {send m e′} x y with <-cmp eid eid′
... | tri< a _ _ = trans (eid<⇒⊏-locally x a) processOrder₁
... | tri> _ _ c = ⊥-elim (1+n≰n (≤-trans y c))
... | tri≈ _ b _ with uniquely-identify {e = e} {e′ = e′} x b
... | refl = processOrder₁
eid<⇒⊏-locally {_} {eid} {e} {_} {suc eid′} {recv _ e′} x y with <-cmp eid eid′
... | tri< a _ _ = trans (eid<⇒⊏-locally x a) processOrder₂
... | tri> _ _ c = ⊥-elim (1+n≰n (≤-trans y c))
... | tri≈ _ b _ with uniquely-identify {e = e} {e′ = e′} x b
... | refl = processOrder₂

⊏-tri-locally {pid} {eid} {_} {pid} {eid′} {_} refl with <-cmp eid eid′
... | tri< a _ _ = inj₁ (eid<⇒⊏-locally refl a)
... | tri≈ _ b _ = inj₂ (inj₁ (uniquely-identify refl b))
... | tri> _ _ c = inj₂ (inj₂ (eid<⇒⊏-locally refl c))
\end{code}

\end{document}
