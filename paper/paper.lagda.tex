\documentclass[acmsmall,review,anonymous]{acmart}
\settopmatter{printfolios=true,printccs=false,printacmref=false}

%% Remove copyright box
\setcopyright{none}
\renewcommand\footnotetextcopyrightpermission[1]{}
\pagestyle{plain}

\bibliographystyle{ACM-Reference-Format}
\citestyle{acmauthoryear}

\usepackage{newunicodechar}
\newunicodechar{ℕ}{\ensuremath{\mathbb{N}}}

\usepackage[references]{agda}

\begin{code}[hide]
module paper where

open import Data.Fin using (Fin)
open import Data.Nat using (ℕ)
\end{code}

\title{Formalizing Vector Clocks in Agda (Functional Pearl)}

\author{Gan Shen}
\affiliation{\institution{University of California, Santa Cruz} \country{USA}}
\author{Simon Guo}
\affiliation{\institution{University of California, Santa Cruz} \country{USA}}
\author{Lindsey Kuper}
\affiliation{\institution{University of California, Santa Cruz} \country{USA}}

\begin{abstract}
Distributed systems are hard to build partly because the lack of
physically synchronous global clocks makes reasoning about causality
sometimes impossible. To this end, vector logical clocks have been
proposed and proved to capture causality, in particular, they preserve
and determin the happens-before relation. However, most of the proofs
are done informally on paper.

In this paper...
\end{abstract}

\begin{document}

\maketitle

\section{Introduction}

\section{System Model}
We model a distributed system as consisting of a fixed number of
processes communicating solely by messages. In Agda, we postulate the
number of processes \AgdaRef{n} as a natural number and the type of
the messages \AgdaRef{Message}.
\begin{code}
postulate
  n : ℕ
  Message : Set
\end{code}
In an abstract way, a process can be viewed as a sequence of events
that take place on it, where events are sendings and receivings of
messages. Similarly, an execution of a distributed system can be
viewed as an collection of sequences of events of each process. We
assume each process is assigned an unique identifier
\AgdaRef{ProcessId}\footnote{Fin n is the Agda type of natural numbers
less than n.} and each message is assigned a local identifier
\AgdaRef{LocalEventId} that corresponds to the order it takes place on
its originating process.
\begin{code}
ProcessId = Fin n
LocalEventId = ℕ
\end{code}

\section{Strong Clock Condition}

\subsection{Happens-Before Preserving}

\subsection{Happens-Before Determining}

\section{Conclusion}
placeholder~\citep{mattern-vector-time, fidge-vector-time, schmuck-dissertation}

\bibliography{refs}

\end{document}
